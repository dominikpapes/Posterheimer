\chapter{Implementacija i korisničko sučelje}
		
		
		\section{Korištene tehnologije i alati}
		
			\textbf{\textit{dio 2. revizije}}
			
			 \textit{Detaljno navesti sve tehnologije i alate koji su primijenjeni pri izradi dokumentacije i aplikacije. Ukratko ih opisati, te navesti njihovo značenje i mjesto primjene. Za svaki navedeni alat i tehnologiju je potrebno \textbf{navesti internet poveznicu} gdje se mogu preuzeti ili više saznati o njima}.
			
			\indent Pri izradi dokumentacije korišten je \textbf{LaTeX}\footnote{https://www.latex-project.org} - označni jezik korišten za uređivanje tekstualnih dokumenata najčešće znanstvene publikacije. Dokument je sastavljen u programu \textbf{TeXstudio}\footnote{https://www.texstudio.org} - uređivaču teksta prilagođen LaTeX-u. UML dijagrami nacrtani su uz pomoć alata \textbf{Astah UML}\footnote{https://astah.net/products/astah-uml}. Dijagrami koji nisu UML tipa nacrtani su u uređivaču \textbf{MS Word}\footnote{https://www.microsoft.com/en/microsoft-365/word}
			
			Udaljeni repozitorij projekta dostupan je na platformi \textbf{GitHub}\footnote{https://github.com}, korištenoj za pohranu svih datoteka potrebnih za rad na projektu.
			
			Za izradu pozadinskog dijela aplikacije korišten je objektno orijentirani programski jezik \textbf{Java}\footnote{https://www.java.com/en} i radni okvir \textbf{Spring Boot}\footnote{https://spring.io/projects/spring-boot} - specijalizacija radnog okvira Spring, s ciljem jednostavnijeg i bržeg oblikovanja web aplikacije. Kao razvojno okruženje korišten je \textbf{IntelliJ IDEA}\footnote{https://www.jetbrains.com/idea}. Tijekom izrade, pozadinski dio aplikacije testiran je pomoću platforme za testiranje API-ja \textbf{Postman}\footnote{https://www.postman.com/}.
			
			Za izradu prednjeg dijela aplikacije korišten je \textbf{React}\footnote{https://react.dev} i JavaScript ekstenzija \textbf{JavaScript XML}\footnote{https://legacy.reactjs.org/docs/introducing-jsx.html}. React, također poznat kao React.js ili ReactJS, je biblioteka u JavaScriptu za izgradnju korisničkih sučelja koju održava \textit{Facebook}. \textit{React} se najčešće koristi kao osnova u razvoju mrežnih ili mobilnih aplikacija. Složene aplikacije u \textit{React}-u obično zahtijevaju korištenje dodatnih biblioteka za interakciju s API-jem.
			
			Baza podataka izrađena je u sustavu za upravljanje bazama podataka \textbf{PostgreSQL}\footnote{https://www.postgresql.org}. Za pristup PostgreSQL sustavu baze podataka korišten je \textbf{pgAdmin 4}\footnote{https://www.pgadmin.org} program. Za izradu ER dijagrama baze podataka korišten je \textbf{ERDPlus}\footnote{https://erdplus.com} alat.
			
			Aplikacija i baza podataka su postavljene na udaljene poslužitelje kao usluga aplikacije \textbf{Render}\footnote{https://render.com}, koja nudi besplatnu opciju pri iznajmljivanju sklopovlja, s ograničenim mogućnostima.
			
			Komunikacija između članova tima ostvarena je putem aplikacija s tom svrhom: \textbf{WhatsApp}\footnote{https://www.whatsapp.com} i \textbf{Discord}\footnote{https://discord.com}.
			
			\eject 
		
	
		\section{Ispitivanje programskog rješenja}
			
			\textbf{\textit{dio 2. revizije}}\\
			
			 \textit{U ovom poglavlju je potrebno opisati provedbu ispitivanja implementiranih funkcionalnosti na razini komponenti i na razini cijelog sustava s prikazom odabranih ispitnih slučajeva. Studenti trebaju ispitati temeljnu funkcionalnost i rubne uvjete.}
			 
			
			\subsection{Ispitivanje komponenti}
			\textit{Potrebno je provesti ispitivanje jedinica (engl. unit testing) nad razredima koji implementiraju temeljne funkcionalnosti. Razraditi \textbf{minimalno 6 ispitnih slučajeva} u kojima će se ispitati redovni slučajevi, rubni uvjeti te izazivanje pogreške (engl. exception throwing). Poželjno je stvoriti i ispitni slučaj koji koristi funkcionalnosti koje nisu implementirane. Potrebno je priložiti izvorni kôd svih ispitnih slučajeva te prikaz rezultata izvođenja ispita u razvojnom okruženju (prolaz/pad ispita). }
			
			Ispitivanje komponenti provedeno je pomoću Selenium WebDrivera\footnote{https://www.selenium.dev/documentation/webdriver/} unutar JUnit 4 testova kao podršku za pisanje ispita unutar programskog jezika Java. Ispitivanje je provedeno nad konferencijom naziva "TestKonferencija". U vrijeme izrade dokumentacije, konferencija se nalazila na drugom mjestu na listi doostupnih konferencija. 
			Izrađeno je 7 ispitnih slučajeva. Uspješnost 6 slučajeve te izazivanje pogreške u jednom prikazni su slikom 5.1.
			
			\begin{figure} [hbt!]
				\includegraphics[width=\linewidth]{Slike/testResults}
				\caption{Prikaz uspješnosti ispitnih slučajeva}
			\end{figure}
			
			U \textbf{prvom ispitnom slučaju} provjerena je funkcionalnost gumba za pristup konferenciji pri incijalnom pristupu pomoću točnog generičkog korisničkog imena (email adrese predviđenu za generički račun) i odgovarajuće lozinke. Predviđen rezultat je dozvoljen pristup stranici konferencije. Ispitni slučaj je uspješan.
			
			
			\begin{lstlisting}
@Test
public void genericLogin() {
	
	System.setProperty("webdriver.chrome.driver", "src\\test\\java\\chromedriver\\chromedriver.exe");
	WebDriver driver = new ChromeDriver();
	driver.manage().timeouts().implicitlyWait(10, TimeUnit.SECONDS);
	
	driver.get("https://posterheimer.onrender.com/");
	org.openqa.selenium.Dimension target = new  org.openqa.selenium.Dimension(1552, 840);
	driver.manage().window().setSize(target);
	
	driver.findElement(By.xpath("(//button[contains(text(),'Pristupi')])[2]")).click();
	// broj postaviti na mjesto konferencije u listi
	
	driver.findElement(By.id("username")).click();
	driver.findElement(By.id("username")).sendKeys("visitor.test@mail.hr");
	driver.findElement(By.id("password")).click();
	driver.findElement(By.id("password")).sendKeys("Lozinka1");
	driver.findElement(By.cssSelector(".btn-primary")).click();
	
	driver.findElement(By.linkText("Konferencija")).click();
	
	String redirURL = driver.getCurrentUrl();
	boolean comperRes = redirURL.contains("/conference");
	if(comperRes == true) {
		System.out.println("Pristupljeno konferenciji");
	} else {
		System.out.println("Error");
	}
	assertEquals(comperRes, true);
	driver.quit();
}
			\end{lstlisting}
			
			U \textbf{drugom ispitnom slučaju} provjerena je funkcionalnost gumba za pristup konferenciji pomoću jedinstvenog korisničkog računa te odjava. Potrebno je upisati točnu email adresu i odgovarajuću lozinku već stvorenog korisničkog računa. Nakon upisa podataka, korisnik dobiva pristup stranici konferencije. Nakon prijave, korisnik se odjavljuje. Predviđen rezultat je povratak na početnu stranicu aplikacije. Ispitni slučaj je uspješan.  

			
			\begin{lstlisting}
@Test
public void correctSpecificLogin() {
	System.setProperty("webdriver.chrome.driver", "src\\test\\java\\chromedriver\\chromedriver.exe");
	WebDriver driver = new ChromeDriver();
	driver.manage().timeouts().implicitlyWait(10, TimeUnit.SECONDS);
	
	driver.get("https://posterheimer.onrender.com/");
	driver.manage().window().setSize(new Dimension(1552, 840));
	driver.findElement(By.cssSelector(".list-group-item:nth-child(4) > .float-end")).click();
	driver.findElement(By.id("username")).click();
	driver.findElement(By.id("username")).sendKeys("register.email@mail.hr");
	driver.findElement(By.id("password")).click();
	driver.findElement(By.id("password")).sendKeys("Lozinka1");
	driver.findElement(By.cssSelector(".btn-primary")).click();
	
	driver.findElement(By.linkText("Konferencija")).click();
	
	String redirURL = driver.getCurrentUrl();
	boolean comperRes = redirURL.contains("/conference");
	if(comperRes == true) {
		System.out.println("Pristupljeno konferenciji");
	} else {
		System.out.println("Error: prijava");
	}
	
	driver.findElement(By.id("user-dropdown")).click();
	driver.findElement(By.cssSelector(".fa-right-from-bracket")).click();
	
	redirURL = driver.getCurrentUrl();
	comperRes = redirURL.equals("https://posterheimer.onrender.com/");
	if(comperRes == true) {
		System.out.println("Ostvarena odjava");
	} else {
		System.out.println("Error: odjava");
	}
	driver.quit();
}	
			\end{lstlisting}
			
			
			U \textbf{trećem ispitnom slučaju} provjerena je funkcionalnost gumba za pristup konferenciji pri neispravnoj prijavi. Predviđen rezultat je prikaz obavijesti "Pogrešan email ili lozinka!" te ostajanje na sučelju za prijavu unutar početne stranice. Ispitni slučaj je uspješan.
			
			\begin{lstlisting}
@Test
public void incorrectLogin() {
	System.setProperty("webdriver.chrome.driver", "src\\test\\java\\chromedriver\\chromedriver.exe");
	WebDriver driver = new ChromeDriver();
	driver.manage().timeouts().implicitlyWait(10, TimeUnit.SECONDS);
	
	driver.get("https://posterheimer.onrender.com/");
	driver.manage().window().setSize(new Dimension(1552, 840));
	driver.findElement(By.xpath("(//button[@type=\'button\'])[2]")).click(); // vrijednost koju mijenjamo

	driver.findElement(By.cssSelector(".mb-3:nth-child(1)")).click();
	driver.findElement(By.id("username")).click();
	driver.findElement(By.id("username")).sendKeys("test");
	driver.findElement(By.id("password")).click();
	driver.findElement(By.id("password")).sendKeys("test");
	driver.findElement(By.cssSelector(".btn-primary")).click();
	
	driver.findElement(By.cssSelector(".alert")).click();

	WebElement alert = driver.findElement(By.cssSelector(".alert"));
	String errorMessage = alert.getText();
	if(errorMessage.contains("email ili lozinka!")) {
		System.out.println("Prikaz prikladne obavijesti");
	}
	
	driver.findElement(By.cssSelector(".btn-close")).click();
	
	String redirURL = driver.getCurrentUrl();
	boolean comperRes = redirURL.equals("https://posterheimer.onrender.com/");
	if(comperRes) {
		System.out.println("Ostvaren rezultat");
	}
	driver.quit();
}
			\end{lstlisting}
			
			Neispravnu prijavu ispitati ćemo i pomoću postojećeg korisničkog računa koji nije namijenjen odabranoj konferenciji. Upisat ćemo redni broj druge konferencije, kao \textit{username} poslati "visitor.test@mail.hr" te "Lozinka1" za \textit{password}. Predviđeni rezultat je ponovan prikaz obavijesti "Pogrešan email ili lozinka!". Ispitni slučaj nije uspješan - obavijest se ne prikazuje te je korisniku dozvoljen pristup konferenciji. Funkcionalnost bi u budućnosti trebalo implementirati.
			
			\begin{figure} [hbt!]
				\includegraphics[width=\linewidth]{Slike/incorrectLoginFail}
				\caption{Prikaz pada ispitnog slučaja}
			\end{figure}
			
			
			
			U \textbf{četvrtom ispitnom slučaju} provjerena je funkcionalnost gumba za pristup konferenciji pri prijavi administratora. Predviđen rezultat je dozvoljen pristup konferenciji te mogućnost dodavanja postera, fotografija i pokrovitelja. Ispitni slučaj je uspješan.
			
			\begin{lstlisting}
	  @Test
public void adminLogin() {
	System.setProperty("webdriver.chrome.driver", "src\\test\\java\\chromedriver\\chromedriver.exe");
	WebDriver driver = new ChromeDriver();
	driver.manage().timeouts().implicitlyWait(20, TimeUnit.SECONDS);
	// Vrijeme povecano zbog ucitavanja postera, koje je znatno sporije za administratora
	
	driver.get("https://posterheimer.onrender.com/");
	driver.manage().window().setSize(new Dimension(1552, 840));
	driver.findElement(By.cssSelector(".list-group-item:nth-child(2) > .float-end")).click();
	driver.findElement(By.id("username")).click();
	driver.findElement(By.id("username")).sendKeys("admin.test@mail.hr");
	driver.findElement(By.id("password")).click();
	driver.findElement(By.id("password")).sendKeys("Lozinka1");
	driver.findElement(By.cssSelector(".btn-primary")).click();
	driver.findElement(By.cssSelector(".conference-content")).click();
	driver.findElement(By.linkText("Posteri")).click();
	driver.findElement(By.cssSelector(".fa-solid")).click();
	driver.findElement(By.cssSelector(".btn-close")).click();
	driver.findElement(By.linkText("Fotografije")).click();
	driver.findElement(By.cssSelector(".image-container > img")).click();
	driver.findElement(By.cssSelector(".modal-title")).click();
	driver.findElement(By.cssSelector(".btn-close")).click();
	driver.findElement(By.linkText("Pokrovitelji")).click();
	driver.findElement(By.cssSelector(".fa-solid")).click();
	driver.findElement(By.cssSelector(".modal-title")).click();
	driver.findElement(By.cssSelector(".btn-close")).click();
	
	String redirURL = driver.getCurrentUrl();
	boolean comperRes = redirURL.contains("sponsors");
	if(comperRes) {
		System.out.println("Ostvaren rezultat");
	}
	driver.quit();
}
			\end{lstlisting}
			
			
	U \textbf{petom ispitnom slučaju} provjerena je funkcionalnost navigacijske trake. Predviđen rezultat je otvaranje stranice konferencije nakon klika na "Konferencija", stranice postera nakon klika na "Posteri", stranice za fotografije nakon klika na "Fotografije" i stranice pokrovitelja nakon klika na "Pokrovitelji". Ispitni slučaj je uspješan.
	
	\begin{lstlisting}
	  @Test
public void navBar() {
	System.setProperty("webdriver.chrome.driver", "src\\test\\java\\chromedriver\\chromedriver.exe");
	WebDriver driver = new ChromeDriver();
	driver.manage().timeouts().implicitlyWait(10, TimeUnit.SECONDS);	
	
	driver.get("https://posterheimer.onrender.com/");
	org.openqa.selenium.Dimension target = new  org.openqa.selenium.Dimension(1552, 840);
	driver.manage().window().setSize(target);
	
	driver.findElement(By.xpath("(//button[contains(text(),'Pristupi')])[2]")).click();
	// broj postaviti na mjesto konferencije u listi
	
	driver.findElement(By.id("username")).click();
	driver.findElement(By.id("username")).sendKeys("visitor.test@mail.hr");
	driver.findElement(By.id("password")).click();
	driver.findElement(By.id("password")).sendKeys("Lozinka1");
	driver.findElement(By.cssSelector(".btn-primary")).click();
	
	driver.findElement(By.linkText("Konferencija")).click();
	String redirURL = driver.getCurrentUrl();
	boolean comperRes = redirURL.contains("/conference");
	if(comperRes == true) {
		System.out.println("Pristupljeno konferenciji");
	} else {
		System.out.println("Error: konferencije");
	}
	
	driver.findElement(By.linkText("Posteri")).click();
	redirURL = driver.getCurrentUrl();
	comperRes = redirURL.contains("/posters");
	if(comperRes == true) {
		System.out.println("Pristupljeno posterima");
	} else {
		System.out.println("Error: posteri");
	}
	
	driver.findElement(By.linkText("Fotografije")).click();
	redirURL = driver.getCurrentUrl();
	comperRes = redirURL.contains("/photos");
	if(comperRes == true) {
		System.out.println("Pristupljeno fotografijama");
	} else {
		System.out.println("Error: fotografije");
	}
	driver.findElement(By.linkText("Pokrovitelji")).click();
	redirURL = driver.getCurrentUrl();
	comperRes = redirURL.contains("/sponsors");
	if(comperRes == true) {
		System.out.println("Pristupljeno pokroviteljima");
	} else {
		System.out.println("Error: pokrovitelji");
	}
	driver.quit();
}
	\end{lstlisting}
	
	U \textbf{šestom ispitnom slučaju} provjerena je funkcionalnost gumba za registraciju pri unosu email adrese netočnog formata. Sva ostala polja su točno ispunjena te je prođen reCAPTCHA test. Predviđen rezultat je neuspješna registracija te obavijest o netočnom unosu. Ispitni slučaj je uspješan.
	Ispitni slučaj je neuspješan u slučaju CAPTCHA Image testa 

			\begin{lstlisting}
	  @Test
public void incorrectRegister() {
	System.setProperty("webdriver.chrome.driver", "src\\test\\java\\chromedriver\\chromedriver.exe");
	WebDriver driver = new ChromeDriver();
	JavascriptExecutor js = (JavascriptExecutor) driver;
	driver.manage().timeouts().implicitlyWait(10, TimeUnit.SECONDS);
	
	driver.get("https://posterheimer.onrender.com/");
	driver.manage().window().setSize(new Dimension(1552, 840));
	driver.findElement(By.cssSelector(".list-group-item:nth-child(2) > .float-end")).click();
	driver.findElement(By.cssSelector("form")).click();
	driver.findElement(By.id("username")).click();
	driver.findElement(By.id("username")).sendKeys("visitor.test@mail.hr");
	driver.findElement(By.id("password")).click();
	driver.findElement(By.id("password")).sendKeys("Lozinka1");
	driver.findElement(By.cssSelector(".btn-primary")).click();
	driver.findElement(By.linkText("Registracija")).click();
	js.executeScript("window.scrollTo(0,0)");
	driver.findElement(By.id("formIme")).click();
	driver.findElement(By.id("formIme")).sendKeys("TestIme");
	driver.findElement(By.id("formPrezime")).click();
	driver.findElement(By.id("formPrezime")).sendKeys("TestPrezime");
	driver.findElement(By.id("formBasicEmail")).click();
	driver.findElement(By.id("formBasicEmail")).sendKeys("email");
	driver.findElement(By.id("formBasicPassword")).click();
	driver.findElement(By.id("formBasicPassword")).sendKeys("lozinka");
	driver.findElement(By.cssSelector(".mb-3:nth-child(2) > #formBasicPassword")).click();
	driver.findElement(By.cssSelector(".mb-3:nth-child(2) > #formBasicPassword")).sendKeys("lozinka");
	driver.switchTo().frame(0);
	driver.findElement(By.cssSelector(".recaptcha-checkbox-border")).click();
	driver.switchTo().defaultContent();
	driver.findElement(By.cssSelector(".ml-2")).click();
	driver.findElement(By.cssSelector(".mx-2 > .mb-3:nth-child(1)")).click();
	driver.findElement(By.cssSelector(".mx-2 > .mb-3:nth-child(1)")).click();
	
	String redirURL = driver.getCurrentUrl();
	boolean comperRes = redirURL.contains("/register");
	if(comperRes == true) {
		System.out.println("Neuspjesna registracija");
	} else {
		System.out.println("Error");
	}
	
	driver.quit();
}	  
			\end{lstlisting}
			
		U \textbf{sedmon ispitnom slučaju} provjerena je funkcionalnost sučelja za dodavanje postera pri unosu datoteke netočnog formata. Prije dodavanje postera potrebna je prijava administratora. Predviđen rezultat je prikaz obavijesti "Izabrana datoteka nije PDF!". Rezultat je ostvaren.  
		
	\begin{lstlisting}
@Test
public void incorrectPosterFormat() {
	System.setProperty("webdriver.chrome.driver", "src\\test\\java\\chromedriver\\chromedriver.exe");
	WebDriver driver = new ChromeDriver();
	JavascriptExecutor js = (JavascriptExecutor) driver;
	File filepath = new File("src\\test\\java\\files\\example.pptx");
	
	
	driver.manage().timeouts().implicitlyWait(10, TimeUnit.SECONDS);
	
	
	driver.get("https://posterheimer.onrender.com/");
	driver.manage().window().setSize(new Dimension(1552, 840));
	driver.findElement(By.cssSelector(".list-group-item:nth-child(2) > .float-end")).click();	  
	
	driver.findElement(By.id("username")).click();
	driver.findElement(By.id("username")).sendKeys("admin.test@mail.hr");
	driver.findElement(By.id("password")).click();
	driver.findElement(By.id("password")).sendKeys("Lozinka1");
	driver.findElement(By.cssSelector(".btn-primary")).click();
	driver.findElement(By.linkText("Posteri")).click();
	driver.findElement(By.cssSelector(".fa-solid")).click();
	
	WebElement fileInput = driver.findElement(By.cssSelector("input[type=file]"));
	fileInput.sendKeys(filepath.getAbsolutePath());
	
	driver.findElement(By.name("imePoster")).click();
	driver.findElement(By.name("imePoster")).sendKeys("Krivi format");
	driver.findElement(By.name("imeAutor")).click();
	driver.findElement(By.name("imeAutor")).sendKeys("Ime");
	driver.findElement(By.name("prezimeAutor")).click();
	driver.findElement(By.name("prezimeAutor")).sendKeys("Prezime");
	driver.findElement(By.cssSelector("form")).click();
	driver.findElement(By.name("posterEmail")).click();
	driver.findElement(By.name("posterEmail")).sendKeys("test.poster@posterheimer.hr");
	driver.findElement(By.cssSelector(".btn")).click();
	
	WebElement alert = driver.findElement(By.cssSelector(".alert"));
	String errorMessage = alert.getText();
	if(errorMessage.contains("Izabrana datoteka nije PDF!")) {
		System.out.println("Prikaz prikladne obavijesti");
	} else {
		System.out.println("Error");
	}		
	
	driver.findElement(By.cssSelector(".alert")).click();
	driver.findElement(By.cssSelector(".btn-close")).click();
	
	
	driver.quit();
}
	\end{lstlisting}
			
			\subsection{Ispitivanje sustava}
			
			 \textit{Potrebno je provesti i opisati ispitivanje sustava koristeći radni okvir Selenium\footnote{\url{https://www.seleniumhq.org/}}. Razraditi \textbf{minimalno 4 ispitna slučaja} u kojima će se ispitati redovni slučajevi, rubni uvjeti te poziv funkcionalnosti koja nije implementirana/izaziva pogrešku kako bi se vidjelo na koji način sustav reagira kada nešto nije u potpunosti ostvareno. Ispitni slučaj se treba sastojati od ulaza (npr. korisničko ime i lozinka), očekivanog izlaza ili rezultata, koraka ispitivanja i dobivenog izlaza ili rezultata.\\ }
			 
			 \textit{Izradu ispitnih slučajeva pomoću radnog okvira Selenium moguće je provesti pomoću jednog od sljedeća dva alata:}
			 \begin{itemize}
			 	\item \textit{dodatak za preglednik \textbf{Selenium IDE} - snimanje korisnikovih akcija radi automatskog ponavljanja ispita	}
			 	\item \textit{\textbf{Selenium WebDriver} - podrška za pisanje ispita u jezicima Java, C\#, PHP koristeći posebno programsko sučelje.}
			 \end{itemize}
		 	\textit{Detalji o korištenju alata Selenium bit će prikazani na posebnom predavanju tijekom semestra.}
		 	
		 	Ispitivanje sustava provedeno je pomoću dodatka za preglednik Selenium IDE\footnote{https://www.selenium.dev/selenium-ide/}. JUnit ispitni slučajevi generirani su pomoću Selenium IDE ispitnih slučajeva.
		 	
		 	\textbf{Prvi ispitni slučaj} provjerava funkcionalnost brisanja korisnika. Prije brisanja korisnika potrebna je prijava registratora. Administrator briše određenog korisnika te se odjavljuje. Uspješnost brisanja korisnika provjeravamo pokušajem prijave izbrisanog korisnika. Očekivani rezultat je neuspješna prijava. Ispitni slučaj je uspješan.
		 	
		 	\begin{figure} [hbt!]
		 		\includegraphics[width=\linewidth]{Slike/deleteUser}
		 		\caption{Prikaz uspješnosti ispitnog slučaja}
		 	\end{figure}
		 	
		\begin{lstlisting}
  @Test
public void deleteUser() {
	driver.get("https://posterheimer.onrender.com/");
	driver.manage().window().setSize(new Dimension(1552, 840));
	driver.findElement(By.cssSelector(".list-group-item:nth-child(5) > .float-end")).click();
	{
		WebElement element = driver.findElement(By.cssSelector(".list-group-item:nth-child(5) > .float-end"));
		Actions builder = new Actions(driver);
		builder.moveToElement(element).perform();
	}
	{
		WebElement element = driver.findElement(By.tagName("body"));
		Actions builder = new Actions(driver);
		builder.moveToElement(element, 0, 0).perform();
	}
	driver.findElement(By.cssSelector(".mb-3:nth-child(1)")).click();
	driver.findElement(By.id("username")).click();
	driver.findElement(By.id("username")).sendKeys("admin.test@mail.hr");
	driver.findElement(By.cssSelector("form")).click();
	driver.findElement(By.id("password")).click();
	driver.findElement(By.id("password")).sendKeys("pass");
	driver.findElement(By.cssSelector(".btn-primary")).click();
	driver.findElement(By.id("user-dropdown")).click();
	driver.findElement(By.linkText("Korisnici")).click();
	driver.findElement(By.cssSelector(".list-group-item:nth-child(16) > .float-start")).click();
	driver.findElement(By.id("root")).click();
	driver.findElement(By.cssSelector(".list-group-item:nth-child(18) > .btn")).click();
	driver.findElement(By.id("user-dropdown")).click();
	driver.findElement(By.linkText("Odjava")).click();
	driver.findElement(By.cssSelector(".list-group-item:nth-child(5) > .float-end")).click();
	{
		WebElement element = driver.findElement(By.cssSelector(".list-group-item:nth-child(5) > .float-end"));
		Actions builder = new Actions(driver);
		builder.moveToElement(element).perform();
	}
	{
		WebElement element = driver.findElement(By.tagName("body"));
		Actions builder = new Actions(driver);
		builder.moveToElement(element, 0, 0).perform();
	}
	driver.findElement(By.id("username")).click();
	driver.findElement(By.id("username")).sendKeys("register2.");
	driver.findElement(By.id("username")).sendKeys("register2.email@mail.hr");
	driver.findElement(By.id("password")).click();
	driver.findElement(By.id("password")).sendKeys("pass");
	driver.findElement(By.cssSelector(".btn-primary")).click();
}
		\end{lstlisting}
		
\textbf{Drugi ispitni slučaj} provjerava funkcionalnost gumba za registraciju pri unosu podataka točnog formata. Tijekom registracije odabrana je opcija "Prikaz lozinke" te prođen CAPTCHA test. Rezultat ispitnog slučaja je uspješan.

\begin{figure} [hbt!]
	\includegraphics[width=\linewidth]{Slike/Register}
	\caption{Prikaz uspješnosti ispitnog slučaja}
\end{figure}

\begin{lstlisting}
	@Test
	public void register() {
		driver.get("https://posterheimer.onrender.com/");
		driver.manage().window().setSize(new Dimension(1552, 840));
		driver.findElement(By.cssSelector(".list-group-item:nth-child(5) > .float-end")).click();
		driver.findElement(By.id("username")).click();
		driver.findElement(By.id("username")).sendKeys("visitor.test@mail.hr");
		driver.findElement(By.id("password")).click();
		driver.findElement(By.id("password")).sendKeys("pass");
		driver.findElement(By.cssSelector(".btn-primary")).click();
		driver.findElement(By.cssSelector(".conference-content")).click();
		driver.findElement(By.cssSelector(".mx-auto:nth-child(1)")).click();
		driver.findElement(By.cssSelector(".card-text")).click();
		driver.findElement(By.linkText("Registracija")).click();
		js.executeScript("window.scrollTo(0,0)");
		driver.findElement(By.id("formIme")).click();
		driver.findElement(By.id("formIme")).sendKeys("TestIme");
		driver.findElement(By.id("formPrezime")).click();
		driver.findElement(By.id("formPrezime")).sendKeys("TestPrezime");
		driver.findElement(By.id("formBasicEmail")).click();
		driver.findElement(By.id("formBasicEmail")).click();
		driver.findElement(By.id("formBasicEmail")).sendKeys("register.email@mail.hr"); // promijeniti email za svaki test
		driver.findElement(By.id("formBasicPassword")).click();
		driver.findElement(By.id("formBasicPassword")).sendKeys("pass");
		driver.findElement(By.cssSelector(".mb-3:nth-child(2) > #formBasicPassword")).click();
		driver.findElement(By.cssSelector(".mb-3:nth-child(2) > #formBasicPassword")).sendKeys("pass");
		driver.findElement(By.cssSelector(".form-check")).click();
		driver.findElement(By.id("formShowPassword")).click();
		driver.findElement(By.id("formShowPassword")).click();
		driver.switchTo().frame(0);
		driver.findElement(By.cssSelector(".recaptcha-checkbox-border")).click();
		driver.switchTo().defaultContent();
		driver.findElement(By.cssSelector(".ml-2")).click();
		assertThat(driver.switchTo().alert().getText(), is("Form submission successful!"));
		js.executeScript("window.scrollTo(0,0)");
		driver.findElement(By.cssSelector("h1")).click();
		driver.findElement(By.cssSelector("h1")).click();
		{
			WebElement element = driver.findElement(By.cssSelector("h1"));
			Actions builder = new Actions(driver);
			builder.doubleClick(element).perform();
		}
		driver.findElement(By.cssSelector("h1")).click();
		driver.findElement(By.cssSelector("h1")).click();
	}
\end{lstlisting}

\textbf{Treći ispitni slučaj} provjerava funkcionalnost gumba za dodavanje konferencije. Prije dodavanja konferencije potrebna je prijava natkorisnika. Podaci vezani za korisnički račun natkorisnika uklonjeni su iz izvornog koda. Predviđen rezultat  je stvaranje nove konferencije. Ispitni slučaj je uspješan.

\begin{figure} [hbt!]
	\includegraphics[width=\linewidth]{Slike/addConferention}
	\caption{Prikaz uspješnosti ispitnog slučaja}
\end{figure}

\begin{lstlisting}
	@Test
	public void addConferention() {
		driver.get("https://posterheimer.onrender.com/");
		driver.manage().window().setSize(new Dimension(1552, 840));
		driver.findElement(By.cssSelector(".fa-solid")).click();
		driver.findElement(By.id("username")).click();
		driver.findElement(By.id("username")).sendKeys("ime natkorisnika");
		driver.findElement(By.id("password")).click();
		driver.findElement(By.id("password")).sendKeys("lozinka natkorisnika");
		driver.findElement(By.cssSelector(".btn-primary")).click();
		driver.findElement(By.cssSelector(".fa-square-plus")).click();
		driver.findElement(By.id("formConferenceName")).click();
		driver.findElement(By.id("formConferenceName")).sendKeys("AutomaticTestKonferencija");
		driver.findElement(By.name("videoUrl")).click();
		driver.findElement(By.name("videoUrl")).sendKeys("https://www.youtube.com/watch?v=ScMzIvxBSi4&ab_channel=BenMarquezTX");
		driver.findElement(By.id("formConferenceCity")).click();
		driver.findElement(By.id("formConferenceCity")).sendKeys("Unska ");
		driver.findElement(By.id("formConferenceCity")).sendKeys("Unska 3");
		driver.findElement(By.id("formConferenceLocation")).click();
		driver.findElement(By.id("formConferenceLocation")).sendKeys("Zagreb");
		driver.findElement(By.id("formConferenceZipCode")).click();
		driver.findElement(By.id("formConferenceZipCode")).sendKeys("10000");
		driver.findElement(By.id("formDateStart")).click();
		driver.findElement(By.id("formDateStart")).sendKeys("2024-01-11T09:13");
		driver.findElement(By.id("formDateEnd")).click();
		driver.findElement(By.id("formDateEnd")).sendKeys("2024-01-14T09:13");
		driver.findElement(By.id("root")).click();
		driver.findElement(By.id("formBasicEmail")).click();
		driver.findElement(By.id("formBasicEmail")).sendKeys("visitor");
		driver.findElement(By.id("formBasicEmail")).sendKeys("visitor.test@mail.hr");
		driver.findElement(By.id("formBasicPassword")).click();
		driver.findElement(By.id("formBasicPassword")).sendKeys("pass");
		driver.findElement(By.cssSelector(".mt-2 > .mb-3:nth-child(2)")).click();
		driver.findElement(By.name("adminUsername")).click();
		driver.findElement(By.name("adminUsername")).sendKeys("admin.test@mail.hr");
		driver.findElement(By.name("adminPassword")).click();
		driver.findElement(By.name("adminPassword")).sendKeys("pass");
		driver.findElement(By.cssSelector(".ml-2")).click();
		js.executeScript("window.scrollTo(0,0)");
	}
\end{lstlisting}
		 	
		 	
		\textbf{Četvrti ispitni slučaj:} prijava registriranog korisnika, odlazak na postere, glasanje za jedan od postera, pokušaj glasanja za drugi poster. Očekivani rezultat je obavijest koja korisniku govori da je nemoguće glasati više od jedanput
			
			\eject 
		
		
		\section{Dijagram razmještaja}
			
			\textbf{\textit{dio 2. revizije}}
			
			 \textit{Potrebno je umetnuti \textbf{specifikacijski} dijagram razmještaja i opisati ga. Moguće je umjesto specifikacijskog dijagrama razmještaja umetnuti dijagram razmještaja instanci, pod uvjetom da taj dijagram bolje opisuje neki važniji dio sustava.}
			 
			 \indent Dijagram razmještaja prikazuje odnos sklopovskih dijelova sustava međusobno i s programskim rješenjima koja su potrebna za korisnikovu interakciju s aplikacijom. Kao dio udaljene poslužiteljske infrastrukture postoje dva poslužiteljska računala: mrežni poslužitelj i poslužitelj baze podataka. Na mrežnom poslužitelju je aktivan proces programa aplikacije koji komunicira s bazom podataka koja je aktivna na vlastitom poslužitelju. Predviđeno je da korisnik koristi mrežni preglednik na vlastitom računalu za komunikaciju s aplikacijom na mrežnom poslužitelju.
			 
			 \begin{figure} [hbt!]
			 	\includegraphics[width=\linewidth]{Slike/DeploymentDiagram}
			 	\caption{Dijagram razmještaja}
			 \end{figure}
			
			\eject 
		
		\section{Upute za puštanje u pogon}
		
			\subsubsection{Kreiranje baze podataka na serveru}
			U ovom projektu je korišten Render oblak. Nakon izrade korisničkog računa možemo pristupiti opciji Dashboard, zatim odabiremo opciju "New +" - PostgreSQL. Unese se ime baze, ostalo se autogenerira, odabire se plan plaćanja (besplatna verzija te različite plaćene) te zatim Create. Time je baza kreirana. Sada ulaskom u dashboard imamo opciju imebaze-db. Klikom na bazu ulazimo u prozor s informacijama o bazi od kojih su nam neke ključne za spajanje baze s backend servisom.
			\newpage
			\begin{figure} [h]
				\includegraphics[width=\linewidth]{Slike/Dashboard-New-PostgreSQL}
				\caption{Dashboard-New-PostgreSQL}
			\end{figure}
			\newpage
			\begin{figure} [h]
				\includegraphics[width=\linewidth]{Slike/Izrada Baze Podataka}
				\caption{Izrada Baze Podataka}
			\end{figure}
			\newpage
			\begin{figure} [h]
				\includegraphics[width=\linewidth]{Slike/Info o Bazi Podataka}
				\caption{Info o Bazi Podataka}
			\end{figure}
			
			\newpage
			\subsubsection{Postavljanje backend servisa na server}
			Prije svega treba pripremiti backend za puštanje na server. Dodaje se Dockerfile u direktorij docker. Sličnim postupkom kao i za bazu, u Dashboard kliknemo "New +", zatim Web Service, nakon povezivanja git računa, dobije se pristup svim repozitorijima kojima račun ima pravo pristupa. Stisnuti connect kraj odgovarajućeg projekta.
			\begin{itemize}
			\item Postaviti ime servisa
			\item Root directory postaviti na IzvorniKod/posterheimer-backend
			\item Environment Docker
			\item Region postaviti na Frankfurt
			\item Na dnu proširiti advanced opciju
			\item Dodati potrebne environment varijable
			\item Stisnuti Create Web Service
			\end{itemize}
			Ako je sve dobro postavljeno, nakon što prođe neko vrijeme dok sve "deploya" na server, servis bi trebao biti "live". 
			\newpage
			\begin{figure} [h]
				\includegraphics[width=\linewidth]{Slike/Dashboard-New-Web Service}
				\caption{Dashboard-New-Web Service }
			\end{figure}
			\newpage
			\begin{figure} [h]
				\includegraphics[width=\linewidth]{Slike/Stvaranje iz git repozitorija}
				\caption{Stvaranje iz git repozitorija}
			\end{figure}
			\newpage
			\begin{figure} [h]
				\includegraphics[width=\linewidth]{Slike/Spajanje s repozitorijem}
				\caption{Spajanje s repozitorijem}
			\end{figure}
			\newpage
			\begin{figure} [h]
				\includegraphics[width=\linewidth]{Slike/Polja za ispunjavanje}
				\caption{Polja za ispunjavanje}
			\end{figure}
			\newpage
			\begin{figure} [h]
				\includegraphics[width=\linewidth]{Slike/Dodavanje Environment varijabli}
				\caption{Dodavanje Environment varijabli}
			\end{figure}
			\newpage
			\begin{figure} [h]
				\includegraphics[width=\linewidth]{Slike/Advance-Docker path}
				\caption{Advance-Docker path}
			\end{figure}
			\newpage
			\begin{figure} [h]
				\includegraphics[width=\linewidth]{Slike/Environment varijable}
				\caption{Environment varijable }
			\end{figure}
			
			\subsubsection{Puštanje frontenda u pogon}
			Prije svega trebamo imati Github profil s repozitorijem koji sadrži naš kod. U ovom primjeru radi se o repozitoriju Posterheimer. Potom se trebamo prijaviti na Render s pripadajućim Github profilom. U Render Dashboardu odabiremo opciju "New" - "Web Service". 
			
			\newpage
			\begin{figure} [h]
			\includegraphics[width=\linewidth]{Slike/Render-Dashboard}
			\caption{Render Dashboard}
			\end{figure}
			
			\newpage
			U sljedećem izborniku odabiremo opciju "Build and deploy from a Git repository"
			
			\begin{figure} [h]
				\includegraphics[width=\linewidth]{Slike/Build-and-deploy}
				\caption{Build and deploy}
			\end{figure}
			
			Iz izbornika izabiremo željeni repozitorij i odaberemo opciju "Connect". U sljedećem izborniku odaberemo željeno ime našeg servisa, regiju u kojoj će se servis pokretati, za europsko tržište odaberemo Frankfurt. Odabiremo granu s koje će se servis pokretati, u našem slučaju to je grana master. Specifično za pokretanje frontend koda želimo da se sve komande pokreću unutar direktorija IzvorniKod/posterheimer-frontend, stoga ćemo to upisati u polje Root directory. Runtime za naš servis je Node. Komanda izgradnje mora biti npm install \&\& npm run build. Kod samog pokretanja servisa treba se izvršiti naredba npm run start koju upisujemo u Start command polje.
			
			\newpage
			\begin{figure} [h]
			\includegraphics[width=\linewidth]{Slike/Run-Commands}
			\caption{Komande za pokretanje}
			\end{figure}
			
			Možemo odabrati vrstu instance našeg servisa, postoji besplatna opcija koja nudi najlošije performanse te koju ćemo koristiti  za potrebe ovog projekta.
			Potrebne su nam 3 varijable okoline, Environment variables.
			
			\begin{itemize}
				\item  API\_BASE\_URL = https://posterheimer-service.onrender.com - središnja ruta za sve API pozive prema serveru
				\item HOST = 0.0.0.0 - IP računala na kojem se izvodi servis
				\item  PORT = 3000 - port na kojem računalo sluša
			\end{itemize}
			
			Nakon što smo unijeli sve vrijednosti možemo kliknuti na "Create Web Service" gumb čime će se započeti puštanje aplikacije u pogon. Nakon kratkog čekanja trebali bismo vidjeti poruku "Your service is live!" nakon čega joj možemo pristupiti putem dodijeljenog linka, u ovom slučaju https://posterheimer-example.onrender.com/.
			
			\newpage
			\begin{figure} [h]
				\includegraphics[width=\linewidth]{Slike/Frontend-Deployed}
				\caption{Aplikacija uspješno pokrenuta u pogon}
			\end{figure}
			\eject 