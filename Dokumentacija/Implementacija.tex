\chapter{Implementacija i korisničko sučelje}
		
		
		\section{Korištene tehnologije i alati}
		
			\textbf{\textit{dio 2. revizije}}
			
			 \textit{Detaljno navesti sve tehnologije i alate koji su primijenjeni pri izradi dokumentacije i aplikacije. Ukratko ih opisati, te navesti njihovo značenje i mjesto primjene. Za svaki navedeni alat i tehnologiju je potrebno \textbf{navesti internet poveznicu} gdje se mogu preuzeti ili više saznati o njima}.
			
			\indent Pri izradi dokumentacije korišten je \textbf{LaTeX}\footnote{https://www.latex-project.org} - označni jezik korišten za uređivanje tekstualnih dokumenata najčešće znanstvene publikacije. Dokument je sastavljen u programu \textbf{TeXstudio}\footnote{https://www.texstudio.org} - uređivaču teksta prilagođen LaTeX-u. UML dijagrami nacrtani su uz pomoć alata \textbf{Astah UML}\footnote{https://astah.net/products/astah-uml}. Dijagrami koji nisu UML tipa nacrtani su u uređivaču \textbf{MS Word}\footnote{https://www.microsoft.com/en/microsoft-365/word}
			
			Udaljeni repozitorij projekta dostupan je na platformi \textbf{GitHub}\footnote{https://github.com}, korištenoj za pohranu svih datoteka potrebnih za rad na projektu.
			
			Za izradu pozadinskog dijela aplikacije korišten je objektno orijentirani programski jezik \textbf{Java}\footnote{https://www.java.com/en} i radni okvir \textbf{Spring Boot}\footnote{https://spring.io/projects/spring-boot} - specijalizacija radnog okvira Spring, s ciljem jednostavnijeg i bržeg oblikovanja web aplikacije. Kao razvojno okruženje korišten je \textbf{IntelliJ IDEA}\footnote{https://www.jetbrains.com/idea}. Tijekom izrade, pozadinski dio aplikacije testiran je pomoću platforme za testiranje API-ja \textbf{Postman}\footnote{https://www.postman.com/}.
			
			Za izradu prednjeg dijela aplikacije korišten je \textbf{React}\footnote{https://react.dev} i JavaScript ekstenzija \textbf{JavaScript XML}\footnote{https://legacy.reactjs.org/docs/introducing-jsx.html}. React, također poznat kao React.js ili ReactJS, je biblioteka u JavaScriptu za izgradnju korisničkih sučelja koju održava \textit{Facebook}. \textit{React} se najčešće koristi kao osnova u razvoju mrežnih ili mobilnih aplikacija. Složene aplikacije u \textit{React}-u obično zahtijevaju korištenje dodatnih biblioteka za interakciju s API-jem.
			
			Baza podataka izrađena je u sustavu za upravljanje bazama podataka \textbf{PostgreSQL}\footnote{https://www.postgresql.org}. Za pristup PostgreSQL sustavu baze podataka korišten je \textbf{pgAdmin 4}\footnote{https://www.pgadmin.org} program. Za izradu ER dijagrama baze podataka korišten je \textbf{ERDPlus}\footnote{https://erdplus.com} alat.
			
			Aplikacija i baza podataka su postavljene na udaljene poslužitelje kao usluga aplikacije \textbf{Render}\footnote{https://render.com}, koja nudi besplatnu opciju pri iznajmljivanju sklopovlja, s ograničenim mogućnostima.
			
			Komunikacija između članova tima ostvarena je putem aplikacija s tom svrhom: \textbf{WhatsApp}\footnote{https://www.whatsapp.com} i \textbf{Discord}\footnote{https://discord.com}.
			
			\eject 
		
	
		\section{Ispitivanje programskog rješenja}
			
			\textbf{\textit{dio 2. revizije}}\\
			
			 \textit{U ovom poglavlju je potrebno opisati provedbu ispitivanja implementiranih funkcionalnosti na razini komponenti i na razini cijelog sustava s prikazom odabranih ispitnih slučajeva. Studenti trebaju ispitati temeljnu funkcionalnost i rubne uvjete.}
			
			\subsection{Ispitivanje komponenti}
			\textit{Potrebno je provesti ispitivanje jedinica (engl. unit testing) nad razredima koji implementiraju temeljne funkcionalnosti. Razraditi \textbf{minimalno 6 ispitnih slučajeva} u kojima će se ispitati redovni slučajevi, rubni uvjeti te izazivanje pogreške (engl. exception throwing). Poželjno je stvoriti i ispitni slučaj koji koristi funkcionalnosti koje nisu implementirane. Potrebno je priložiti izvorni kôd svih ispitnih slučajeva te prikaz rezultata izvođenja ispita u razvojnom okruženju (prolaz/pad ispita). }
			
			U \textbf{prvom ispitnom slučaju} provjerena je funkcionalnost gumba za pristup konferenciji pri incijalnom pristupu pomoću točnog generičkog korisničkog imena i odgovarajuće lozinke. Predviđen rezultat je dozvoljen pristup stranici konferencije na kojoj je dostupna stranica postera, ali stranice fotografija i pokrovitelja nude prijavu tj. registraciju. Ispitni slučaj je uspješan (Slika 5.1).
			
			\begin{figure} [hbt!]
				\includegraphics[width=\linewidth]{Slike/genericLoginWithPagesLog}
				\caption{Prikaz uspješnosti ispitnog slučaja}
			\end{figure}
			
			
			\begin{lstlisting}
  @Test
public void genericLoginWithPages() {
	driver.get("https://posterheimer.onrender.com/");
	driver.manage().window().setSize(new Dimension(1552, 840));
	driver.findElement(By.cssSelector(".list-group-item:nth-child(5) > .float-end")).click();
	driver.findElement(By.cssSelector(".mb-3:nth-child(1)")).click();
	driver.findElement(By.id("username")).click();
	driver.findElement(By.id("username")).sendKeys("visitor.test@mail.hr");
	driver.findElement(By.cssSelector(".mb-3:nth-child(2)")).click();
	driver.findElement(By.id("password")).click();
	driver.findElement(By.id("password")).click();
	driver.findElement(By.id("password")).sendKeys("pass");
	driver.findElement(By.cssSelector(".btn-primary")).click();
	driver.findElement(By.linkText("Posteri")).click();
	driver.findElement(By.linkText("Fotografije")).click();
	driver.findElement(By.linkText("Pokrovitelji")).click();
	driver.findElement(By.cssSelector(".login-prompt")).click();
}


			\end{lstlisting}
			
			U \textbf{drugom ispitnom slučaju} provjerena je funkcionalnost gumba za pristup konferenciji pomoću jedinstvenog korisničkog računa pomoću točne email adrese i odgovarajuće lozinke. Nakon prijave, korisnik se odjavljuje. Ispitni slučaj je uspješan. 
			\begin{figure} [hbt!]
				\includegraphics[width=\linewidth]{Slike/CorrectSpecificLogin}
				\caption{Prikaz uspješnosti ispitnog slučaja}
			\end{figure}
			
			\begin{lstlisting}
  @Test
public void correctSpecificLogin() {
	driver.get("https://posterheimer.onrender.com/");
	driver.manage().window().setSize(new Dimension(1552, 840));
	driver.findElement(By.cssSelector(".list-group-item:nth-child(5) > .float-end")).click();
	driver.findElement(By.id("username")).click();
	driver.findElement(By.id("username")).sendKeys("test.email@mail.hr");
	driver.findElement(By.id("password")).click();
	driver.findElement(By.id("password")).sendKeys("pass");
	driver.findElement(By.cssSelector(".btn-primary")).click();
	driver.findElement(By.id("user-dropdown")).click();
	driver.findElement(By.cssSelector(".fa-right-from-bracket")).click();
}
			\end{lstlisting}
			
			
			U \textbf{trećem ispitnom slučaju} provjerena je funkcionalnost gumba za pristup konferenciji pri neispravnoj prijavi. Predviđen rezultat je prikaz obavijesti "Pogrešan email ili lozinka!" te ostajanje na sučelje za prijavu. Ispitni slučaj je uspješan.
			\begin{figure} [hbt!]
				\includegraphics[width=\linewidth]{Slike/IncorrectLogin}
				\caption{Prikaz uspješnosti ispitnog slučaja}
			\end{figure}
			
			\begin{lstlisting}
@Test
public void incorrectLogin() {
	driver.get("https://posterheimer.onrender.com/");
	driver.manage().window().setSize(new Dimension(1552, 840));
	driver.findElement(By.cssSelector(".list-group-item:nth-child(5) > .float-end")).click();
	{
		WebElement element = driver.findElement(By.cssSelector(".list-group-item:nth-child(5) > .float-end"));
		Actions builder = new Actions(driver);
		builder.moveToElement(element).perform();
	}
	{
		WebElement element = driver.findElement(By.tagName("body"));
		Actions builder = new Actions(driver);
		builder.moveToElement(element, 0, 0).perform();
	}
	driver.findElement(By.cssSelector(".mb-3:nth-child(1)")).click();
	driver.findElement(By.id("username")).click();
	driver.findElement(By.id("username")).sendKeys("test");
	driver.findElement(By.id("password")).click();
	driver.findElement(By.id("password")).sendKeys("test");
	driver.findElement(By.cssSelector(".btn-primary")).click();
	driver.findElement(By.cssSelector(".alert")).click();
	driver.findElement(By.cssSelector(".alert")).click();
	driver.findElement(By.cssSelector(".alert")).click();
	{
		WebElement element = driver.findElement(By.cssSelector(".alert"));
		Actions builder = new Actions(driver);
		builder.doubleClick(element).perform();
	}
	driver.findElement(By.cssSelector(".alert")).click();
	driver.findElement(By.cssSelector(".alert")).click();
	driver.findElement(By.cssSelector(".alert")).click();
	driver.findElement(By.cssSelector(".btn-close")).click();
}
			\end{lstlisting}
			
			
\textbf{Četvrti ispitni slučaj} provjerava funkcionalnost gumba za registraciju pri unosu email adrese točnog formata. Predviđen rezultat je uspješna registracija.
			\begin{lstlisting}
	
			\end{lstlisting}

\textbf{Peti ispitni slučaj} provjerava funkcionalnost gumba za registraciju pri unosu email adrese netočnog formata. Predviđen rezultat je neuspješna registracija.		
			\begin{lstlisting}
	
			\end{lstlisting}
			
\textbf{Šesti ispitni slučaj} provjerava funkcionalnost navigacijske trake. Predviđen rezultat je da će nas određeni gumb voditi na stranicu istog naziva.

			\begin{lstlisting}
	
			\end{lstlisting}

\textbf{Sedmi ispitni slučaj} provjerava funkcionalnost gumba za dodavanje konferencija/postera.
			
			
			\subsection{Ispitivanje sustava}
			
			 \textit{Potrebno je provesti i opisati ispitivanje sustava koristeći radni okvir Selenium\footnote{\url{https://www.seleniumhq.org/}}. Razraditi \textbf{minimalno 4 ispitna slučaja} u kojima će se ispitati redovni slučajevi, rubni uvjeti te poziv funkcionalnosti koja nije implementirana/izaziva pogrešku kako bi se vidjelo na koji način sustav reagira kada nešto nije u potpunosti ostvareno. Ispitni slučaj se treba sastojati od ulaza (npr. korisničko ime i lozinka), očekivanog izlaza ili rezultata, koraka ispitivanja i dobivenog izlaza ili rezultata.\\ }
			 
			 \textit{Izradu ispitnih slučajeva pomoću radnog okvira Selenium moguće je provesti pomoću jednog od sljedeća dva alata:}
			 \begin{itemize}
			 	\item \textit{dodatak za preglednik \textbf{Selenium IDE} - snimanje korisnikovih akcija radi automatskog ponavljanja ispita	}
			 	\item \textit{\textbf{Selenium WebDriver} - podrška za pisanje ispita u jezicima Java, C\#, PHP koristeći posebno programsko sučelje.}
			 \end{itemize}
		 	\textit{Detalji o korištenju alata Selenium bit će prikazani na posebnom predavanju tijekom semestra.}
		 	
		 	\textbf{Prvi ispitni slučaj:} prijava administratora, brisanje registriranog korisnika, odjava, pokušaj prijave izbrisanog korisnika. Očekivani rezultat je neuspješna prijava
		 	\textbf{Drugi ispitni slučaj:} prijava administratora, postavljanje postera u krivom formatu. Očekivani rezultat je greška
		 	\textbf{Treći ispitni slučaj:} prijava registriranog korisnika, odlazak na stranicu pokrovitelja, klik na logo pokrovitelja. Očekivani rezultat je odlazak na stranicu pokrovitelja
		 	\textbf{Četvrti ispitni slučaj:} prijava registriranog korisnika, odlazak na postere, glasanje za jedan od postera, pokušaj glasanja za drugi poster. Očekivani rezultat je obavijest koja korisniku govori da je nemoguće glasati više od jedanput
			
			\eject 
		
		
		\section{Dijagram razmještaja}
			
			\textbf{\textit{dio 2. revizije}}
			
			 \textit{Potrebno je umetnuti \textbf{specifikacijski} dijagram razmještaja i opisati ga. Moguće je umjesto specifikacijskog dijagrama razmještaja umetnuti dijagram razmještaja instanci, pod uvjetom da taj dijagram bolje opisuje neki važniji dio sustava.}
			 
			 \indent Dijagram razmještaja prikazuje odnos sklopovskih dijelova sustava međusobno i s programskim rješenjima koja su potrebna za korisnikovu interakciju s aplikacijom. Kao dio udaljene poslužiteljske infrastrukture postoje dva poslužiteljska računala: mrežni poslužitelj i poslužitelj baze podataka. Na mrežnom poslužitelju je aktivan proces programa aplikacije koji komunicira s bazom podataka koja je aktivna na vlastitom poslužitelju. Predviđeno je da korisnik koristi mrežni preglednik na vlastitom računalu za komunikaciju s aplikacijom na mrežnom poslužitelju.
			 
			 \begin{figure} [hbt!]
			 	\includegraphics[width=\linewidth]{Slike/DeploymentDiagram}
			 	\caption{Dijagram razmještaja}
			 \end{figure}
			
			\eject 
		
		\section{Upute za puštanje u pogon}
		
			\textbf{\textit{dio 2. revizije}}\\
		
			 \textit{U ovom poglavlju potrebno je dati upute za puštanje u pogon (engl. deployment) ostvarene aplikacije. Na primjer, za web aplikacije, opisati postupak kojim se od izvornog kôda dolazi do potpuno postavljene baze podataka i poslužitelja koji odgovara na upite korisnika. Za mobilnu aplikaciju, postupak kojim se aplikacija izgradi, te postavi na neku od trgovina. Za stolnu (engl. desktop) aplikaciju, postupak kojim se aplikacija instalira na računalo. Ukoliko mobilne i stolne aplikacije komuniciraju s poslužiteljem i/ili bazom podataka, opisati i postupak njihovog postavljanja. Pri izradi uputa preporučuje se \textbf{naglasiti korake instalacije uporabom natuknica} te koristiti što je više moguće \textbf{slike ekrana} (engl. screenshots) kako bi upute bile jasne i jednostavne za slijediti.}
			
			
			 \textit{Dovršenu aplikaciju potrebno je pokrenuti na javno dostupnom poslužitelju. Studentima se preporuča korištenje neke od sljedećih besplatnih usluga: \href{https://aws.amazon.com/}{Amazon AWS}, \href{https://azure.microsoft.com/en-us/}{Microsoft Azure} ili \href{https://www.heroku.com/}{Heroku}. Mobilne aplikacije trebaju biti objavljene na F-Droid, Google Play ili Amazon App trgovini.}
			
			
			\eject 