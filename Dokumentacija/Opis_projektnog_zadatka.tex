\chapter{Opis projektnog zadatka}
		
		Cilj projekta je razviti programsku podršku za stvaranje web aplikacije „Posterheimer“ koja će olakšati pregled radova sudionicima stručnih konferencija prikazom svakog prijavljenog rada odgovarajućim posterom/prezentacijom unutar pregledne galerije. Omogućit će glasovanje putem aplikacije, automatsko rangiranje autora i njihovih radova, te pregled dodatnih informacija o konferenciji kao što su datum i vrijeme njezinog održavanja, prikaz mjesta održavanja pomoću karte i vremenske prognoze za tu lokaciju, video prijenos konferencije, pregled fotografija s konferencije i pregled pokrovitelja konferencije.
		
		Prilikom pokretanja aplikacije prikazuje se popis aktivnih konferencija. Prije pristupa određenoj konferenciji prikazuje se zahtjev za prijavu u sustav. Inicijalni pristup moguće je ostvariti pomoću generičkog korisničkog računa. Za prijavu tog tipa potrebno je generičko korisničko ime (adresa e-pošte predviđena za određenu konferenciju) i odgovarajuća generička lozinka. Spomenuti podaci bit će dostupni samo sudionicima konferencije koji će ih dobiti prilikom dolaska i prijave na samu konferenciju. Podaci vezani uz generički račun zajednički su za sve posjetitelje konferencije. Nakon što su pristupili konferenciji korištenjem generičkog korisničkog računa, korisnici mogu izraditi vlastiti korisnički račun pomoću opcije registracije. Nakon registracije, korisnici za pristup konferenciji (prijavu) mogu koristiti vlastiti korisnički račun. 

		U nastavku dokumenta, za korisnike koji su za pristup konferenciji koristili samo generički korisnički račun koristiti će se naziv "korisnici posjetitelji" ili kraće "posjetitelji". Korisnici koji još nisu posjetili konferenciju pomoću generičkog računa zvat će se "neregistrirani korisnici". Korisnici koji se registriraju, odnosno izrade svoj korisnički račun, zvat će se "registrirani korisnici". 
		
		Svaki posjetitelj ima pristup osnovnim podacima o samoj konferenciji (vrijeme i mjesto održavanja, kao i relevantnu vremensku prognozu, te prikaz navedene lokacije na karti). Osim toga, može pregledavati postere/prezentacije i izraditi vlastiti korisnički račun.
		
		\newpage
		
		Za kreiranje novog računa potrebni su sljedeći podaci:
		\begin{packed_item}
			\item adresa e-pošte
			\item lozinka
			\item ime
			\item prezime
		\end{packed_item}
		
		Registracijom u sustav korisniku se dodjeljuju prava "registriranog korisnika". 
		Registrirani korisnici nakon prijave zadržavaju sva prava koja ima korisnik posjetitelj te uz njih dobivaju i neke nove mogućnosti:
		\begin{packed_item}
			\item glasovanje za jedan od ponuđenih postera
			\item praćenje trenutnih događanja u glavnoj konferencijskoj dvorani pomoću video prijenosa
			\item pregled i preuzimanje fotografija fotografiranih za vrijeme trajanja konferencije
			\item pregled pokrovitelja konferencije
			\item pregled konačnih rezultata glasovanja kada ono završi
		\end{packed_item}
		
		Glasovanje je moguće samo tijekom određenog razdoblja koje je određeno danima i vremenom održavanja konferencije, odnosno ono će završiti dva dana prije završetka same konferencije. Nakon završenog postupka glasovanja, objavljuju se rezultati koji su dostupni svim registriranim korisnicima. Registrirani korisnici mogu dati glas samo jednom posteru te svoj glas ne mogu naknadno promijeniti. Glasovanje nije obavezno, dakle korisnici se mogu suzdržati od glasovanja.  
		
		Događanja na konferenciji će se prenositi uživo video prijenosom putem usluge "YouTube". Eventualne fotografije konferencije koje bi administrator objavio na stranici mogu pristupiti svi registrirani korisnici, koji ih zatim mogu pregledati u uvećanom izdanju ili preuzeti na svoj uređaj.
		
		Autori postera ili prezentacija koji sudjeluju na stručnom skupu elektroničkom poštom dostavljaju sve potrebne materijale administratoru zaduženog za konferenciju proizvoljnom metodom izvan sklopa aplikacije (npr. elektroničkom poštom), a nakon završetka postupka glasovanja primaju obavijest o rangu svojeg rada prema glasovima posjetitelja. Porukom elektroničke pošte se prva tri nagrađena rada poziva na dodjelu nagrade. Također, sve se sudionike elektroničkom poštom obavještava o mjestu i vremenu dodjele nagrade. Rezultate glasanja mogu vidjeti samo registrirani korisnici pritiskom na prikladni gumb.
		
		\newpage
		
		Uz korisnika posjetitelja i registriranog korisnika postoje još dvije vrste korisnika, a to su:
		\begin{packed_item}
			\item administrator
			\item natkorisnik
		\end{packed_item}
		
		$\underline{Administrator}$ konferencije zadužen je za upravljanje svim podacima vezanih uz konferenciju. Svaka konferencija ima vlastitog administratora. On ima pristup bazi s popisom registriranih korisnika i njihovim podacima te može brisati korisnike registrirane na konferenciji za koju je zadužen. Ne može upravljati registriranim korisnicima ostalih konferencija. Obavlja prijavu autora, radova i postera i ima mogućnost ažuriranja tih podataka. Dostupni su mu svi podaci te može definirati sve potrebne uvjete za ispravan rad sustava.
		
		$\underline{Natkorisnik}$ je korisnik koji postoji u bazi podataka od samog postavljanja aplikacije na mrežu i ima ovlasti stvaranja i brisanja konferencija, te prilikom stvaranja nove konferencije registrira administratora zaduženog za nju. Prilikom stvaranja nove konferencije podaci koje je natkorisnik dužan unijeti koji služe kao njezin kontekst su:
		\begin{packed_item}
			\item ime
			\item poveznica na video prijenos
			\item adresa lokala gdje se održava
			\item mjesto u kojem se održava
			\item poštanski broj navedenog mjesta
			\item vrijeme početka
			\item vrijeme završetka
			\item generička adresa e-pošte za prijavu posjetitelja
			\item generička lozinka za prijavu posjetitelja
			\item adresa e-pošte zaduženog administratora
			\item lozinka za zaduženog administratora
		\end{packed_item}
		
		Aplikacija je građena za specifičnu uporabu stvaranja konferencija natjecateljskog tipa, čime je namjera zadovoljiti potrošačku nišu koju druge slične aplikacije poput "b2match.com", "Wix.com" i "monday.com" ne ciljaju. Također, cilj aplikacije je privući klijente iznimnom jednostavnošću dizajna i funkcionalnosti, što znači da ju je lako koristiti i da pruža efikasan i brz način za organiziranje željenog događaja.

		\newpage
		
		Neke od mogućih nadogradnja aplikacije su: 
		\begin{packed_item}
			\item omogućiti administratoru estetsku konfiguraciju i personalizaciju stranice same konferencije, čime bi se olakšalo organizatorima da na vizualan način plasiraju svoju poruku, što bi bilo ostvareno jednostavnim sučeljem na kojem bi administrator mogao birati boje elemenata stranice ili postavljati vlastite fotografije ili logotipe u prednji ili zadnji plan prozora
			\item prikaz više opcija o lokaciji održavanja konferencije, na primjer dodati  interaktivnu kartu mjesta konferencije s označenim važnim lokacijama ili integrirati informacije o lokalnim restoranima, hotelima i prijevozu
			\item omogućiti registriranim korisnicima komentiranje postera
			\item implementirati dvofaktorsku ovjeru autentičnosti kako bi se dodatno osigurao pristup sustavu
			\item razvoj mobilne aplikacije za Android i iOS uređaje
			\item uvesti analitičke alate za praćenje korištenja platforme tijekom konferencije i generiranje izvještaja o učinkovitosti i angažmanu korisnika nakon završetka konferencije
		\end{packed_item}

		\eject
		
