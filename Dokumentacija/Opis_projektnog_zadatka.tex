\chapter{Opis projektnog zadatka}
		
		\textbf{\textit{dio 1. revizije}}\\
		
		\textit{Na osnovi projektnog zadatka detaljno opisati korisničke zahtjeve. Što jasnije opisati cilj projektnog zadatka, razraditi problematiku zadatka, dodati nove aspekte problema i potencijalnih rješenja. Očekuje se minimalno 3, a poželjno 4-5 stranica opisa. Teme koje treba dodatno razraditi u ovom poglavlju su:}
		\begin{packed_item}
			\item \textit{potencijalna korist ovog projekta}
			\item \textit{postojeća slična rješenja (istražiti i ukratko opisati razlike u odnosu na zadani zadatak). Dodajte slike koja predočavaju slična rješenja.}
			\item \textit{skup korisnika koji bi mogao biti zainteresiran za ostvareno rješenje.}
			\item \textit{mogućnost prilagodbe rješenja }
			\item \textit{opseg projektnog zadatka}
			\item \textit{moguće nadogradnje projektnog zadatka}
		\end{packed_item}
		
		Cilj projekta je razviti programsku podršku za stvaranje web aplikacije „Posterheimer“ koja će olakšati pregled radova sudionicima stručne konferencije prikazom svakog rada i/ili izlaganja odgovarajućim posterom ili prezentacijom. Omogućit će glasanje preko aplikacije te automatsko rangiranje autora i njihovih radova te pregled dodatnih informacija o konferenciji kao što su datum i vrijeme održavanja konferencije, vremenska prognoza za područje održavanja konferencije, video prijenos konferencije, pregled fotografija s konferencije i pregled pokrovitelja konferencije.
		
		Prilikom pokretanja aplikacije prikazuje se zahtjev za prijavu u sustav generičkom lozinkom uz opciju registracije ili prijavu s postojećim registriranim računom.
		Korisnici koji se registriraju, odnosno izrade svoj korisnički račun, zvat će se "registrirani korisnici" ili "prijavljeni korisnici", a korisnici koji koriste generički račun zvat će se "korisnici posjetitelji" ili "neprijavljeni korisnici". Korisnik se može prijaviti s računom korisnika posjetitelja koji ima generičku lozinku koja je svim posjetiteljima konferencije zajednička. Svi posjetitelji dobivaju korisničko ime Posjetitelj. Svaki posjetitelj može pregledavati postere ili prezentacije i izraditi svoj korisnički račun.
		
		Za kreiranje novog računa potrebni su sljedeći podaci:
		\begin{packed_item}
			\item adresa e-pošte
			\item korisničko ime
			\item lozinka
			\item ime
			\item prezime
		\end{packed_item}
		
		Registracijom u sustav korisniku se dodjeljuju prava "registriranog korisnika". Registrirani korisnik može pregledati i mijenjati sljedeće osobne podatke:
		\begin{packed_item}
			\item ime
			\item prezime
			\item lozinka
		\end{packed_item}

		Registrirani korisnici nakon prijave zadržavaju sva prava koja ima korisnik posjetitelj te uz njih dobivaju i neke nove mogućnosti:
		\begin{packed_item}
			\item Glasanja za jedan od postera
			\item Praćenje trenutnih događanja u glavnoj konferencijskoj dvorani pomoću video prijenosa
			\item Pregled i preuzimanje fotografija fotografiranih za vrijeme trajanja konferencije
			\item Pregled pokrovitelja konferencije klikom u izborniku na pokrovitelje konferencije
			\item Pregled informacija o vremenu i mjestu održavanja konferencije
			\item Pregled vremenske prognoze za mjesto u kojem se održava konferencija
			\item Pregled konačnih rezultata glasanja jednom kad ono završi
		\end{packed_item}
		
		Glasanje je moguće samo tijekom određenog razdoblja koje je određeno danima i vremenom održavanja konferencije. Nakon završenog postupka glasanja, objavljuju se rezultati koji su dostupni svim registriranim korisnicima. Glasati mogu samo registrirani korisnici te mogu glasati za najviše jedan poster. Sve dok traje glasanje glasači mogu maknuti svoj glas s postera za kojeg su već glasali i dodijeliti ga nekom drugom posteru ili ga ne dodijeliti nikome.
		
		Događanja s konferencije bit će uživo prenošena pomoću video prijenosa preko YouTube servisa. Svi događaji se fotografiraju i fotografije se spremaju u galeriju fotografija kojoj mogu pristupiti svi registrirani korisnici. Fotografije korisnici mogu pregledavati i  spremati lokalno na svoj uređaj.
		
		Registriranim korisnicima je dostupan dio s informacijama o mjestu održavanja konferencije koji sadrži podatke o trenutnim vremenskim uvjetima i vremenskoj prognozi za navedenu lokaciju.
		
		Uz korisnika posjetitelja i registriranog korisnika postoje još 2 vrste korisnika, a to su
		\begin{packed_item}
			\item autor rada
			\item administrator
		\end{packed_item}
		
		$\underline{Autor}$ koji sudjeluje na stručnom skupu elektroničkom poštom dostavljaju sve potrebne materijale sistemskom administratoru. Autor prima obavijest o rangu svojeg rada prema glasovima posjetitelja. Porukom elektroničke pošte ih se poziva na dodjelu nagrade za prva tri nagrađena rada. Sve sudionike se elektroničkom poštom obavještava o mjestu i vremenu dodjele nagrade. Rezultate glasanja mogu vidjeti samo registrirani korisnici odabirom opcije rezultati glasanja u izborniku.
		
		$\underline{Administrator}$ sustava ima najveće ovlasti. On ima pristup bazi s popisom registriranih korisnika i njihovim podacima te može brisati registrirane korisnike. Obavlja prijavu autora, radova i postera. Ima mogućnost ažuriranja tih podataka. Određuje vrijeme u kojem je glasanje moguće. Dostupni su mu svi podaci te može definirati sve potrebne uvjete za ispravan rad sustava.

		\eject
		
