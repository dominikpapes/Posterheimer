\chapter{Arhitektura i dizajn sustava}		
Arhitektura je podijeljena u tri podsustava:

\begin{itemize}
	\item Web poslužitelj
	\item Web aplikacija
	\item Baza podataka
\end{itemize}

\indent Internetski preglednik služi za pregled web-stranica i njihovog višemedijskog sadržaja, a one komuniciraju s poslužiteljima slanjem zahtjeva i primanjem odgovora. Preglednik ima sposobnost interpretacije koda kojim je pisana stranica u ljudski čitljiv oblik.
\newline
\indent Web poslužitelj je posrednik između korisnika i aplikacije, te služi kao osnova rada aplikacije. Komunikacija se ostvaruje korisničkim HTTP (engl. \textit{Hyper Text Transfer Protocol}) zahtjevima koji obično u zaglavlju imaju definiranu GET ili POST metodu za dohvaćanje odnosno predaju podataka. Poslužitelj na njih odgovara dostavom traženog sadržaja.
\newline
\indent Web aplikacija po nalogu poslužitelja pristupa ili mijenja podatke iz baze podataka i vraća HTML dokument koji se potom prikazuje korisniku u sučelju preglednika.
Programski jezik koji smo odabrali za razvoj pozadinskog dijela aplikacije je "Java" zajedno sa "Spring" radnim okvirom, a za dizajn korisničkog sučelja odabrali smo "JavaScript", s radnim okvirom "React". Odabrano razvojno okružje je "IntelliJ IDEA". Arhitektura sustava će se temeljiti na stilističkoj varijaciji arhitekture zasnovane na događajima (engl. \textit{event based architecture}) - MVC obrazcu. Spring radni okvir podržava MVC (Model-View-Controller) obrazac, te kao takav ima gotove predloške koji nam olakšavaju razvoj web aplikacije.
\newline

\clearpage
\indent MVC obrazac omogućuje odvojen razvoj navedenih slojeva aplikacije što znatno olakšava ispitivanje, razvijanje i dodavanje novih svojstava u sustav.
\newline
Dijelovi MVC obrazca su:
\begin{itemize}
	\item Model(\textit{Model}) - rješava problem interakcije korisnika s bazom podataka, predstavlja bazu i služi za komunikaciju upravljača s bazom, te je na taj način zadužen za logiku vezanu za podatke i njihov prijenos
	\item Prikaz(\textit{View}) - prikaz podataka i njihovih reprezentacija na korisničkom sučelju
	\item Upravljač(\textit{Controller}) - povezuje komponente modela i prikaza i služi kao posrednik između tih komponenata i korisnika, a sam nije zadužen za obradu podataka, međudjeluje s modelom kako bi dohvatio podatke i s prikazom kako bi ih prikazao korisniku
\end{itemize}

\begin{figure} [hbt!]
	\includegraphics[width=\linewidth]{Slike/ArhitekturaSustava}
	\caption{Reprezentacija arhitekture sustava}
\end{figure}

		\clearpage

		\section{Baza podataka}
			
		Za potrebe našeg sustava koristiti ćemo relacijsku bazu podataka koja svojom strukturom olakšava modeliranje stvarnog svijeta. Gradivna jedinka baze je relacija, odnosno tablica koja je definirana svojim imenom i skupom atributa. Zadaća baze podataka je brza i jednostavna pohrana, izmjena i dohvat podataka za daljnju obradu.
		Baza podataka ove aplikacije sastoji se od sljedećih entiteta: 
		
		\begin{itemize}
			\item Konferencija
			\item Korisnik
			\item Poster
			\item Pokrovitelj
			\item Pokrovitelj-Konferencije
			\item Fotografija
		\end{itemize}
		
		\clearpage
		
		\subsection{Opis tablica}
	
	\noindent\textbf{Konferencija } Ovaj entitet sadržava sve važne informacije o stručnoj konferenciji. Sadrži atribute: Identifikator konferencije, ime konferencije, datum i vrijeme početka konferencije, datum i vrijeme završetka konferencije, mjesto održavanja konferencije, generičko korisničko ime i lozinku koja se koristi za pristup konferenciji. Ovaj entitet u vezi je \textit{One-to-Many} s entitetom Korisnik preko identifikatora konferencije, u vezi \textit{One-to-Many} s entitetom Poster preko identifikatora konferencije, u vezi \textit{One-to-Many} s entitetom Fotografija preko identifikatora konferencije i u vezi \textit{Many-to-Many} s entitetom Pokrovitelj preko identifikatora konferencije. 
	
	\begin{longtblr}[
		label=none,
		entry=none
		]{
			width = \textwidth,
			colspec={|X[6,l]|X[6, l]|X[20, l]|}, 
			rowhead = 1,
		} %definicija širine tablice, širine stupaca, poravnanje i broja redaka naslova tablice
		\hline \SetCell[c=3]{c}{\textbf{Konferencija}}	 \\ \hline[3pt]
		\SetCell{LightGreen}id\_konferencija & INT	&  	jedinstveni brojčani identifikator konferencije  	\\ \hline
		ime\_ konferencija	& VARCHAR &   jedinstveno ime konferencije	\\ \hline 
		datum\_vrijeme \_pocetka & TIMESTAMP & datum i vrijeme početka konferencije  \\ \hline
		datum\_vrijeme \_završetka & TIMESTAMP & datum i vrijeme završetka konferencije \\ \hline
		mjesto	& VARCHAR & ime mjesta u kojem se održava konferencija \\ \hline 
		generic\_ username	& VARCHAR & generičko korisničko ime koje koriste svi posjetitelji konferencije prije izrade vlastitog korisničkog računa  \\ \hline 
		generic\_ password & VARCHAR & hash lozinka generičkog korisničkog računa \\ \hline
	\end{longtblr}
	
	\clearpage
	
	\noindent\textbf{Korisnik } Ovaj entitet sadržava sve važne informacije o korisniku aplikacije. Sadrži atribute: E-mail adresu korisnika, lozinku, ime, prezime, oznaku ima li korisnik ovlasti administratora, oznaku radi li se o generičkom računu, identifikator konferencije na kojoj korisnik sudjeluje i identifikator postera za koji je korisnik glasao.  Ovaj entitet je u vezi \textit{Many-to-One} s entitetom Konferencija preko identifikatora konferencije i u vezi \textit{Many-to-One} s entitetom Poster preko identifikatora postera. 
	
	\begin{longtblr}[
		label=none,
		entry=none
		]{
			width = \textwidth,
			colspec={|X[6,l]|X[6, l]|X[20, l]|}, 
			rowhead = 1,
		} %definicija širine tablice, širine stupaca, poravnanje i broja redaka naslova tablice
		\hline \SetCell[c=3]{c}{\textbf{Korisnik}}	 \\ \hline[3pt]
		\SetCell{LightGreen}email & VARCHAR	&  e-mail adresa služi kao jedinstveni identifikator korisnika 	\\ \hline
		lozinka	& VARCHAR & hash lozinka  	\\ \hline 
		ime & VARCHAR & ime korisnika  \\ \hline 
		prezime & VARCHAR	& prezime korisnika 		\\ \hline 
		admin & BOOLEAN &  oznaka ima li ovlasti administratora		\\ \hline 
		visitor & BOOLEAN &  oznaka radi li se o generičkom računu (TRUE), kada posjetitelj napravi vlastiti korisnički račun vrijednost varijable postaje FALSE	\\ \hline 
		\SetCell{LightBlue} id\_konferencija	& INT & jedinstveni identifikator konferencije na kojoj korisnik sudjeluje (konferencija.id\_konferencija)  	\\ \hline 
		\SetCell{LightBlue} glasao\_za & VARCHAR & jedinstveni identifikator postera za koji je korisnik glasao (poster.imePoster), ukoliko korisnik nije glasao vrijednost atributa je NULL \\ \hline
	\end{longtblr}
	
	\clearpage
	
	\noindent \textbf{Poster } Ovaj entitet sadržava sve važne informacije o radu i/ili izlaganju koje je prikazano odgovarajućim posterom ili prezentacijom. Sadrži atribute: Ime postera, put do datoteke postera, ime i prezime autora, e-mail adresu kojom je autor poslao poster i identifikator konferencije na kojoj se rad izlaže. Ovaj entitet je u vezi \textit{Many-to-One} s entitetom Konferencija preko identifikatora konferencije i u vezi \textit{One-to-Many} s entitetom Korisnik preko identifikatora postera. 
	
	
	\begin{longtblr}[
		label=none,
		entry=none
		]{
			width = \textwidth,
			colspec={|X[6,l]|X[6, l]|X[20, l]|}, 
			rowhead = 1,
		} %definicija širine tablice, širine stupaca, poravnanje i broja redaka naslova tablice
		\hline \SetCell[c=3]{c}{\textbf{Poster}}	 \\ \hline[3pt]
		\SetCell{LightGreen}ime\_poster & VARCHAR	&  ime postera služi kao jedinstveni identifikator	\\ \hline
		file\_path & VARCHAR & put do PDF datoteke postera ili prezentacije  \\ \hline 
		ime\_autor & VARCHAR & ime autora \\ \hline
		prezime\_autor & VARCHAR & prezime autora \\ \hline
		poster\_email & VARCHAR & e-mail adresa autora \\ \hline
		\SetCell{LightBlue} id\_konferencija	& INT & jedinstveni brojčani identifikator konferencije kojoj poster pripada (konferencija.id\_konferencija)  	\\ \hline 
	\end{longtblr}
	
	\clearpage
	
	\noindent \textbf{Pokrovitelj } Ovaj entitet sadržava sve važne informacije o pokrovitelju. Sadrži atribute: Ime pokrovitelja i promotivne materijale. Ovaj entitet je u vezi \textit{Many-to-Many} s entitetom Konferencija.
	
	
	\begin{longtblr}[
		label=none,
		entry=none
		]{
			width = \textwidth,
			colspec={|X[6,l]|X[6, l]|X[20, l]|}, 
			rowhead = 1,
		} %definicija širine tablice, širine stupaca, poravnanje i broja redaka naslova tablice
		\hline \SetCell[c=3]{c}{\textbf{Pokrovitelj}}	 \\ \hline[3pt]
		\SetCell{LightGreen}ime\_ pokrovitelja & VARCHAR & jedinstveni identifikator pokrovitelja  	\\ \hline
		promotivni\_ materijal & VARCHAR & put do datoteke, promotivni materijal je logotip pokrovitelja   \\ \hline 
	\end{longtblr}
	
	\noindent \textbf{Pokrovitelj-Konferencije } Ovaj entitet sadržava sve informacije o odnosu konferencije i pokrovitelja odnosno kojoj konferenciji pokrovitelj pripada.
	
	
	\begin{longtblr}[
		label=none,
		entry=none
		]{
			width = \textwidth,
			colspec={|X[6,l]|X[6, l]|X[20, l]|}, 
			rowhead = 1,
		} %definicija širine tablice, širine stupaca, poravnanje i broja redaka naslova tablice
		\hline \SetCell[c=3]{c}{\textbf{Pokrovitelj-Konferencije}}	 \\ \hline[3pt]
		\SetCell{LightBlue} ime\_
		pokrovitelja	& VARCHAR & jedinstveni identifikator pokrovitelja (pokrovitelj.imePokrovitelja)     	\\ \hline
		\SetCell{LightBlue}id\_konferencija	& INT & jedinstveni brojčani identifikator konferencije (konferencija.id\_konferencija)  	\\ \hline 
		
	\end{longtblr}
	
	
	\noindent \textbf{Fotografija } Ovaj entitet sadržava sve informacije o fotografijama s konferencije. Sadrži atribute: Identifikator fotografije, sliku fotografije i identifikator konferencije koja je fotografirana. Ovaj entitet u vezi je \textit{Many-to-One} s entitetom Konferencija. 
	
	
	\begin{longtblr}[
		label=none,
		entry=none
		]{
			width = \textwidth,
			colspec={|X[6,l]|X[6, l]|X[20, l]|}, 
			rowhead = 1,
		} %definicija širine tablice, širine stupaca, poravnanje i broja redaka naslova tablice
		\hline \SetCell[c=3]{c}{\textbf{Fotografije}}	 \\ \hline[3pt]
		\SetCell{LightGreen}id & INT & jedinstveni brojčani identifikator fotografije   	\\ \hline
		file\_path	& VARCHAR &  put do fotografije	\\ \hline 
		\SetCell{LightBlue} id\_konferencija	& INT & jedinstveni brojčani identifikator konferencije (konferencija.id\_konferencija)  	\\ \hline 
	\end{longtblr}
	
	\clearpage
			
			\subsection{Dijagram baze podataka}
				
					\begin{figure} [h]
						\includegraphics[width=\linewidth]{Slike/ERDijagram}
						\caption{E-R dijagram baze podataka}
					\end{figure}
			
			\eject
			
			
		\section{Dijagram razreda}
			
			Slike 4.3, 4.4 i 4.5 prikazuju dijagrame razreda pozadinskog dijela MVC arhitekture. Slike 4.3 i 4.4 prikazuju upravljačke razrede, te DAO (\textit{Data Access Object} razrede. 
			
			Dijagrami su zbog preglednosti podijeljeni po slojevima, odnosno na pojedinom dijagramu prikazane su samo ovisnosti između razreda istog MVC sloja, a prema nazivima samih razreda mogu se logički povezati funkcionalnosti među slojevima.
			
			Razredi dijagrama "Modeli" preslikavaju entitete i atribute baze podataka. Razred Korisnik predstavlja općeniti model korisnika aplikacije koji može biti posjetitelj konferencije ili administrator. Može se registrirati u sustav unoseći osnove informacije. Razred Konferencija predstavlja skup podataka koji su potrebni za registraciju konferencije i koji se prikazuju korisnicima. Razred Poster predstavlja skup podataka koji su potrebni za dodavanje postera i koji se prikazuju korisnicima. Razred Fotografija predstavlja skup podataka vezan uz pojedinačnu fotografiju konferencije. Razred Pokrovitelj predstavlja pokrovitelja konferencije.
			
			\begin{figure} [hbt!]
				\includegraphics[width=\linewidth]{Slike/ClassDiagramControllerPremaKodu}
				\caption{Dijagram razreda - Upravljači}
			\end{figure}
			
			\begin{figure} [hbt!]
				\includegraphics[width=\linewidth]{Slike/ClassDiagramRepositoryPremaKodu}
				\caption{Dijagram razreda - DAO}
			\end{figure}
			
			\begin{figure} [hbt!]
				\includegraphics[width=\linewidth]{Slike/ClassDiagramModelPremaKodu}
				\caption{Dijagram razreda - Modeli}
			\end{figure}
			
			\clearpage
			 
			\textbf{\textit{dio 2. revizije}}\\			
			
			\textit{Prilikom druge predaje projekta dijagram razreda i opisi moraju odgovarati stvarnom stanju implementacije}
			
			
			
			\eject
		
		\section{Dijagram stanja}
			
			
			\textbf{\textit{dio 2. revizije}}\\
			
			\textit{Potrebno je priložiti dijagram stanja i opisati ga. Dovoljan je jedan dijagram stanja koji prikazuje \textbf{značajan dio funkcionalnosti} sustava. Na primjer, stanja korisničkog sučelja i tijek korištenja neke ključne funkcionalnosti jesu značajan dio sustava, a registracija i prijava nisu. }
			
			
			\eject 
		
		\section{Dijagram aktivnosti}
			
			\textbf{\textit{dio 2. revizije}}\\
			
			 \textit{Potrebno je priložiti dijagram aktivnosti s pripadajućim opisom. Dijagram aktivnosti treba prikazivati značajan dio sustava.}
			
			\eject
		\section{Dijagram komponenti}
		
			\textbf{\textit{dio 2. revizije}}\\
		
			 \textit{Potrebno je priložiti dijagram komponenti s pripadajućim opisom. Dijagram komponenti treba prikazivati strukturu cijele aplikacije.}