\chapter{Zaključak i budući rad}
		 
		 Cilj projekta bio je razviti upotrebljivu aplikaciju od koje bi koristi imali organizatori događanja kao što su skupovi ili konvencije natjecateljskog tipa, gdje natjecatelji predstavljaju svoj rad posterom ili prezentacijom, a oni prisutni na događanju svojim glasom doprinose odluci o pobjedniku.
		 
		 Tim koji je radio na aplikaciji sastojao se od šest članova, a rad na projektu trajao je sedamnaest tjedana, podijeljenih na dva dijela.
		 
		 Završetak prvog dijela obilježilo je postavljanje potrebnih poslužitelja, te na njima uspostava početne baze podataka i početne inačice aplikacije. Prije toga se provelo prikupljanje i analiza zahtjeva, osmišljavanje funkcionalnosti aplikacije i dizajn arhitekture sustava, što je uključivalo odabir radnih okruženja i alata. Glavnina vremena ove etape iskoristila se na detaljan ispis obrazaca uporabe i crtanje dijagrama na temelju kojih bi se kasnije napisao kod.
		 
		 S početkom drugog dijela razvoja, tim se podijelio u tri para, pri čemu su dva člana bila zadužena za prednji \textit{(engl. frontend)}, a dva za pozadinski \textit{(engl. backend)} dio koda, te dva člana za dovršavanje dokumentacije. U ovoj fazi se uglavnom stjecalo potrebno znanje i usavršavale se vještine za rad s odabranim radnim okvirima i knjižnicama, koje se tada koristilo za oblikovanje koda prisutnog u konačnom rješenju. Po uzoru na to rješenje pisao se nastavak dokumentacije, odnosno njezini dijelovi koji ovaj put služe kako bi opisali funkcionalnost programirane aplikacije. Razvoj prednjeg dijela sustava na početku se vršio neovisno od pozadinskog, kao i ispitivanje. Kad su oba dijela bili zasebno funkcionalni, povezalo ih se u cjelinu nakon čega je krenulo konačno ispitivanje i dorada, kao i intenzivniji rad na dokumentiranju funkcionalnosti napisanog koda.
		 
		 Komunikacija unutar parova bila je konstantno uspostavljena putem ranije spomenutih usluga za udaljenu komunikaciju, a prema potrebi se sazvala nekolicina timskih sastanaka, koji su bili relativno učestali tijekom prve faze razvoja, a sukladno s izmijenjenom metodom rada u drugoj fazi puno rjeđi.
		 
		 Tijekom rada na aplikaciji svi članovi su nailazili na tehničke izazove vezane uglavnom uz neiskustvo u korištenju odabranih tehnologija i alata, zbog čega je značajan dio vremena trošio na njihovo proučavanje i rješavanje grešaka nastalih u kodu. Znanja stečena radom na projektu u korištenju radnih okvira za razvoj poslužiteljske strane aplikacije i dizajn korisničkog sučelja, metodama i standardima implementiranja njihove komunikacije, pa čak i radom s alatima za pisanje valjane dokumentacije, kao i samo iskustvo rada u timu pokazat će se korisnima u budućim projektima ili nastavku razvijanja ovog.
		 
		 Planirane funkcionalnosti koje nisu implementirane u konačnoj inačici aplikacije u vrijeme pisanja dokumenta su:
		 \begin{itemize}
		 	\item Ograničenje djelokruga posjetitelja i administratora na vlastitu konferenciju. Posjetiteljski i administratorski računi definirani prilikom stvaranja neke konferencije se mogu iskoristiti za uspješan pristup bilo kojoj konferenciji.
		 	\item Provjera pripada li korisnik konferenciji. Nakon što posjetitelj stvori račun unutar jedne konferencije, može s tim računom pristupiti bilo kojoj drugoj konferenciji.
		 	\item Notifikacije unutar okvira aplikacije. Kada sustav šalje poruku e-pošte kao obavijest o rezultatu glasovanja ili poziv na dodjelu nagrade, ista notifikacija se neće pojaviti korisniku na mrežnoj stranici.
		 \end{itemize}
		
		\eject 