\chapter{Zaključak i budući rad}
		
		\textbf{\textit{dio 2. revizije}}\\
		
		 \textit{U ovom poglavlju potrebno je napisati osvrt na vrijeme izrade projektnog zadatka, koji su tehnički izazovi prepoznati, jesu li riješeni ili kako bi mogli biti riješeni, koja su znanja stečena pri izradi projekta, koja bi znanja bila posebno potrebna za brže i kvalitetnije ostvarenje projekta i koje bi bile perspektive za nastavak rada u projektnoj grupi.}
		
		 \textit{Potrebno je točno popisati funkcionalnosti koje nisu implementirane u ostvarenoj aplikaciji.}
		 
		 Cilj projekta bio je razviti upotrebljivu aplikaciju od koje bi koristi imali organizatori događanja kao što su skupovi ili konvencije natjecateljskog tipa, gdje natjecatelji predstavljaju svoj rad posterom ili prezentacijom, a oni prisutni na događanju svojim glasom doprinose odluci o pobjedniku.
		 Tim koji je radio na aplikaciji sastojao se od šest članova, a rad na projektu trajao je sedamnaest tjedana, podijeljenih na dva dijela.
		 Završetak prvog dijela obilježilo je postavljanje potrebnih poslužitelja, te na njima uspostava početne baze podataka i početne inačice aplikacije. Prije toga se provelo prikupljanje i analiza zahtjeva, osmišljavanje funkcionalnosti aplikacije i dizajn arhitekture sustava, što je uključivalo odabir radnih okruženja i alata. Glavnina vremena ove etape iskoristila se na detaljan ispis obrazaca uporabe i crtanje dijagrama na temelju kojih bi se kasnije napisao kod.
		 S početkom drugog dijela razvoja, tim se podijelio u tri para, pri čemu su dva člana bila zadužena za prednji \textit{(engl. frontend)}, a dva za pozadinski \textit{(engl. backend)} dio koda, te dva člana za dovršavanje dokumentacije. U ovoj fazi se uglavnom stjecalo potrebno znanje i usavršavale se vještine za rad s odabranim radnim okvirima i knjižnicama, koje se tada koristilo za oblikovanje koda prisutnog u konačnom rješenju. Po uzoru na to rješenje pisao se nastavak dokumentacija, odnosno njezini dijelovi koji ovaj put služe kako bi opisali funkcionalnost programirane aplikacije.
		
		\eject 